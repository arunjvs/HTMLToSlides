\subsection{Slides Generation}

The third step in the slide generation process is generation of slides using 
the content returned by the summarizing step. The summarizing step summarizes 
the content of the paper and returns the important sentences to be included in the slides.
The Generator takes these sentences along with the images and references, which these
sentences are referring to, and creates slides using them.

The number of slides will depend on the total number of characters in the sentences selected.
We ensure that there is atleast one dedicated slide for each section and subsection.
The order of slides is same as the order of sections/subsections in the paper.
All the slides are assigned same title as their corresponding section title.
If there is a slide having a sentence which is referring to any graphical element, then
corresponding element will be displayed in the slide next to this. And if in a slide,
there are lines referring to other documents/links, then these references are displayed
at the bottom of the same slide. An overview of how the lines are included in the
slides is presented in the Algorithm \ref{slide_generation}.

\begin{algorithm}[H]
 \SetLine % For v3.9
 \KwData{selectedLinesSet, title }
 \KwResult{sequenceOfSlides }
 \caption{Slide Generation}\label{slide_generation}
 $n$ = maximum number of characters per slide\\
 $linesConsideredSet$ = []\\
 $imagesSet$ = []\\
 $referencesSet$ = []\\
 \For{line in selectedLinesSet}{
 	\If{len(linesConsideredSet + line) $>$ n}{
 		Create a $slide$\\
 		Set the $title$ of the $slide$\\
 		Add the $linesConsideredSet$ to the $slide$\\
 		Add the $referencesSet$ at the bottom of the $slide$\\
 		Append the $slide$ to the output presentation\\
 	\For{image in imagesSet}{
 		Create a $slide$ for the image\\
 		Append the $slide$ to the output presentation\\
	} 	
 	}
 	$linesConsidered \leftarrow line$\\
 	$referencesSet \leftarrow $ references, the line citing to\\
 	$imagesSet \leftarrow $ images, the line referring to
 }
 \If{linesConsideredSet is not Empty}{
 	create a $slide$ for the lines remaining in the $linesConsideredSet$\\
 	Append the $slide$ to the output presentation
 }
\end{algorithm}
