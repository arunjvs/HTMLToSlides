\subsection{Summarization}

The second step in the slide generation process is that of summarization of the text
as present in the complete research paper. The parsing step returns the structure of
the paper in our predefined XML format, with all the lines lines, sections, subsections
in the hierarchical order, and the objective of this stage is to select only those lines
that are most informative, and would be most relevant for a presentation. We applied 
different heuristic techniques to summarize different parts of the paper, as discussed
in following sections.

\subsubsection{Summarizing Introduction}
Introduction is an important section that introduces the problem, the motivation to solving it
and gives an overall direction towards the solution, along with discussion an overview of
the contributions. Abstract of the paper also summarizes the paper, albeit in a more succinct 
format. Hence, we use the approach as proposed in \cite{sravanthi}. For each sentence 
in introduction, we compute it's TF-IDF based cosine similarity with the complete abstract
text, and select the the top $n$ sentences with highest scores.

\subsubsection{Summarizing Model}
We consider the central part of the paper, with the approach, results, analysis etc.
to be the model part of the paper, that contains the bulk of information content.
The information itself may be divided into multiple subsections and bulleted points,
and may contain images, tables, mathematical proofs/expressions etc.

Based on our experiments, we observed that the sections tree structure as extracted
by the parser is an important part of the paper, and we include each of the sections
and subsections in the final set of slides in the same hierarchy as detected
in original paper. However, we summarize the text content in each of those sections
recursively. We use the following set of heuristics for summarizing lines in each:
\begin{enumerate}
	\item \textbf{Bulleted Points} Some sections in the papers itself contain a set of
	points in bulleted fashion, or with numbering. Such points might also be spread
	across with text in between, but have continuous numbering. In such a case, we
	only take those points to be representative text for that section.
	\item \textbf{Other Text} In case the section does not contain bullets, we manually
	select a subset of the lines from that section. We determine a score for every line
	by computing its TF-IDF score with the set of keywords relevant to the paper. 
	Lines with frequent occurance of keywords is usually more relevant, and 
	we tend to select such sentences.
	\item \textbf{Boosting image references} Since images are an important 
	part of presentations, we boost the score of lines with image references.
	That increases the possibility of that line getting included, and hence 
	of the associated image getting included in the slides as well.
\end{enumerate}

\paragraph{Keyword Set Expansion}
The model section summarization depends heavily on the set of keywords, which we
mostly extract from the keywords section of the paper itself. However, this
set is usually very small, and sometimes ineffective for selection of best
subset of sentences, hence we propose an algorithm to expand that set 
using the paper itself. We use co-occurrence probabilities to compute the
other potential keywords, according to Algorithm \ref{algo_set_expand}. 

\begin{algorithm}[H]\label{algo_set_expand}
 \SetLine % For v3.9
 \KwData{allLinesSet, keywordsSet }
 \KwResult{expandedKeywordsSet }
 pairs = []
 \For{line in lines}{
  \If{line contains any keyword}{
    \For{token in line.tokenized}{
	  $pairs \leftarrow (token, keyword)$
	}
  }
  \For{pair in pairs}{
	\If{frequency > threshold}{
		$expandedKeywordsSet \leftarrow pair.token$
	}  
  }
 }
 \caption{Keyword Set Expansion}
\end{algorithm}

\subsubsection{Summarizing Conclusion}

\subsection{Selecting images, references}

\newpage
\vspace{40mm}