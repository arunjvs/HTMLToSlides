
\documentclass[conference, compsoc]{IEEEtran}
% Add the compsoc option for Computer Society conferences.





% correct bad hyphenation here
\hyphenation{op-tical net-works semi-conduc-tor}


\begin{document}
%
% paper title
% can use linebreaks \\ within to get better formatting as desired
\title{Generating slides from HTML pages}


% author names and affiliations
% use a multiple column layout for up to three different
% affiliations
\author{\IEEEauthorblockN{Arun JVS}
\IEEEauthorblockA{JVS roll no}
\and
\IEEEauthorblockN{Rohit Girdhar}
\IEEEauthorblockA{201001047}
\and
\IEEEauthorblockN{Sudheer Kumar}
\IEEEauthorblockA{201001149}}


% make the title area
\maketitle


\begin{abstract}
%\boldmath
The abstract goes here.
\end{abstract}
% This preserves the distinction between vectors and scalars. However,
% if the conference you are submitting to favors bold math in the abstract,
% then you can use LaTeX's standard command \boldmath at the very start
% of the abstract to achieve this. Many IEEE journals/conferences frown on
% math in the abstract anyway.


\section{Introduction}
% no \IEEEPARstart
This demo file is intended to serve as a ``starter file''
for IEEE conference papers produced under \LaTeX\ using
IEEEtran.cls version 1.7 and later.
% You must have at least 2 lines in the paragraph with the drop letter
% (should never be an issue)
I wish you the best of success. \cite{test}

\section{Related Work}


{\small
\bibliographystyle{plain}
\bibliography{ref}
}

\end{document}


