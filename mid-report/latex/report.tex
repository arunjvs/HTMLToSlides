
\documentclass[conference]{IEEEtran}
% Add the compsoc option for Computer Society conferences.


% correct bad hyphenation here
\hyphenation{op-tical net-works semi-conduc-tor}


\begin{document}
%
% paper title
% can use linebreaks \\ within to get better formatting as desired
\title{Generating slides from HTML pages}


% author names and affiliations
% use a multiple column layout for up to three different
% affiliations
\author{\IEEEauthorblockN{Arun JVS}
\IEEEauthorblockA{201001079}
\and
\IEEEauthorblockN{Rohit Girdhar}
\IEEEauthorblockA{201001047}
\and
\IEEEauthorblockN{Sudheer Kumar}
\IEEEauthorblockA{201001149}}


% make the title area
\maketitle


%\begin{abstract}
%\boldmath
%The abstract goes here.
%\end{abstract}
% This preserves the distinction between vectors and scalars. However,
% if the conference you are submitting to favors bold math in the abstract,
% then you can use LaTeX's standard command \boldmath at the very start
% of the abstract to achieve this. Many IEEE journals/conferences frown on
% math in the abstract anyway.


%\section{Introduction}
% no \IEEEPARstart
%This demo file is intended to serve as a ``starter file''
%for IEEE conference papers produced under \LaTeX\ using
%IEEEtran.cls version 1.7 and later.
% You must have at least 2 lines in the paragraph with the drop letter
% (should never be an issue)
%I wish you the best of success.

\section{Motivation}
Presenting information using slides is an effective way for audience retention.
Presenters take the aid of slides for presenting research papers in conferences, for
explaining them in classrooms and for other academic purposes. But creating slides from
the research papers is a time consuming task for the presenter. So, it would be a great
benefit for them if there is an easy way for generating the slides from the research
papers. Also, sometimes people want to go through the research papers just to understand
the overview of them. Slides provide an easier way for them to understand the overview of
the topic. This gives rise to need for a tool which automatically generates slides from
research papers. Automatic generation of slides from research paper is an easier task
compared to generation of slides from some random document because research papers have
an almost similar structure and we can address the problem of automatic generation of slides
from research papers by exploiting the structure of the research papers. Once this problem
is addressed, this work can be easily extended to automatically generate slides from other
structured entities like book chapters etc.

\section{Problem Definition}
Given an HTML version of a research paper which is in accordance with the conference/journal
proceedings, a slideshow is created which contains an summarized version of the paper.
This Slideshow will contain the important points, tables, graphs, figures which helps to
explain the paper. This may not serve as a final presentation but provides a good starting
point for preparing it.

\section{Related Work}
There has been limited work that addresses the problem of automatic slides generation.
Previous works include approaches like section-wise summarization \cite{sravanthi},
alignment methods for matching document regions with presentation regions \cite{brandon},
extracting topics and itemized elaborations from tagged documents.
Many of these systems use multiple ideas from Information Retrieval, such as TF-IDF weights,
query expansion, query-specific summarization, POS-tagging, etc..
Some of them are purely generative, while others learn models from an existing corpus of
document-presentation pairs. These systems are tuned to work with different input formats
such as PDF, XML, \LaTeX\ and PPT, each of which preserve a varying amount of semantic and
structural information about the original text. In the following two subsections, we review
two papers \cite{sravanthi},\cite{brandon} which particularly deal with technical papers.

\subsection{SlidesGen}
The first work we review is by Sravanthi et al. \cite{sravanthi}. They propose an
novel framework for automatic generation of presentation slides for technical papers.
They take \LaTeX\ documents of research papers as input, and return the presentation slides.
Their method depends on the assumption that by and large, conference papers have a similar structure:
an abstract, followed by sections that can be broadly classified as introduction, related work,
actual work (model), experiments, conclusion/results and bibliography.
A slide in their system contains a title and some bulleted points that are important in 
that section. They evaluate the system by surveying the response of people who use their system.

Their system is divided into multiple stages. The first stage is pre-processing stage.
In this step, the \LaTeX\ documents are converted into XML using a public domain converter
LaTeXML. 

Next, the generate what they call "configuration file", which contains configuration
parameters for each section (since each section has a different point of view and writing style).
This involves categorization of the section and extraction of key phrases from the section
which are then stored in the configuration file.
For example, a section with large number of cite tags, or with title containing words such as
"related work" or "literature survey" is categorized as related works section.

The next step is of extracting key phrases.	Most research papers have associated key phrases 
that can be help categorize the content in paper, and contain important concepts introduced
in the paper. They are mostly related to model and experiments section, and can be used to 
summarize the same. So, the keywords given at the beginning are added to keyphrases for those
sections in the configuration file. Also, few other phrases as the names of subsections etc 
are also added to the configuration file.

Next, they use QueSTS summarizer to summarize the model, experiment and conclusions section.
QueSTS represents the text as a Integrated Graph where sentence is a node, and edge exists 
between 2 sentences if the sentences are similar. The edge weight is defined as the 
cosine similarity between the 2 sentences (above a given minimum threshold)
Given the keyphrases computed previously, they are tokenized and 
a centrality based (to the query phrase) node weight is computed for the node corresponding to each token.
Thus, they get query specific node weights and query independent edge weights.

From the above graph, they construct a Contextual Tree for each term $q$ and node $r$.
All the trees at node $r$ are merged into a Summary Graph at $r$. The SGraphs generated
from each node are ranked using a scoring model, and the one with best rank is returned as
the summary.

The final step of this procedure is slide generation from the XML and configuration file.
The introduction slides are generated by comparing introduction sentence with the abstract 
using consine similarity and ones with high similarity are placed in the slides.
They give a similar method for generating slides from related work section as well.
From model and experiments section, the slides are generated using the above QueSTS
mechanism. Similarly, conclusion slides are generated using comparison with keywords
such as "proposed", "concluded" etc.

Another important aspect in slides is graphics. Graphics are added along with sentences 
that either refer to it or is present along with it.

The paper finally discusses the issues in alignment of sentences and generation of slides.
They also evaluate their slides manually with multiple users, taking their feedback.
Most users gave a satisfaction level of more than 8/10 to the slides generated.

\subsection{Alignment Methods}
This paper by Brandon et. al. \cite{brandon} takes a different approach to slide generation,
inspired by the human throught process while making a slide out of a paper.
They focus their efforts on \textit{alignment} methods - first breaking up the document and
presentation into regions and then performing a matching on them.
\textit{Slide regions} include bullets, headings, and other text spans, while \textit{Paper regions}
include paragraphs, section headings, and list items.

They generate their corpus of 296 paper-presentation pairs from workshops of technical conferences,
through simple searching. The papers were PDF format, and presentation were a mixture of PDF and PPT.
Before working with them, they convert them into custom XML formats, which represent relevant parts of
the original data as logical regions (orthographic boundaries). They prefer such physical regions over
semantic regions for its simplicity to implement and verify.

The alignment problem now reduces to an IR query, where the query is a slide region (which is the
section/subsection the slide is related to), and the documents are the target regions from the paper.
They compare two TF-IDF based scoring methods, with and without query expansion, resulting in 4
alignment methods.

The procedure of scoring is as follows. For each token in each region TF-IDF is computed, where TF is
the frequency of the token in the region and DF is the number of regions containing the token's stem.
The slide region is tokenized and POS tagged to remove non-content words. Each token in the query is
stemmed and then may or may not be query-expanded depending on the method. A score is then calculated
for each target region with the query. Two scoring methods were used - one uses the average TF-IDF
score of the search terms relative to the target region, and the other uses the quantity of the matched
terms.

The evaluation of their methods gives critical insights. First, a vast majority of the slide regions are
not alignable (zero score with all target regions) - meaning that a lot of information in slides is not
present in the paper - contrary to their hypothesis. Then they define an \textit{alignable} accuracy against
the \textit{raw} accuracy, considering only alignable slide regions. They find that the best algorithm
on an average gives an average 75\% alignable accuracy, but only 50\% raw accuracy. Query expansion seems
to have little or negative impact on the aligners and that the second scoring method is better than
the first.

After more results, they conclude that the data indicates that the task of presentation generation
is highly dependent on the end purpose the presentation will serve, as well as the target audience and other
factors. Also query expansion generally degraded performance, possibly because authors tend to use wording in
slides similar to their paper, and that using synonyms for query expansion is not aggressive enough, and may
require hypernyms, immediate hyponyms, and other semantically related terms. Also a possible loss of accuracy
could be polysemy of words, causing query expansion to be incorrect and insensitive to context.

\section{Method}
An outline of our method is as follows. It's divided into the following stages:
\subsection{Parsing and Preprocessing}
	We convert the HTML DOM tree of the research paper into a standard XML with specific tags for
	abstract, sections etc.
	Since most of the pages in the dataset follow a fixed schema with definite text and tags,
	we can easily create such a representation of the paper. The schema is similar to as in
	\cite{sravanthi}.

\subsection{Slides Generation}
	We use an approach similar to \cite{sravanthi}. Our final set of slides will consist of
	pre-defined set of sections, such as Introduction, Related Work, Model (method), Experiments,
	Results and Conclusion, which are common in most technical papers. 

	As per \cite{sravanthi}, we use the QueSTS based summarization approach for the central part of
	the paper, consisting of the methods proposed, experiments and results. This is because this constitutes
	the bulk of the paper, and simple keyword based retrieval might not extract all the 
	relevant information. The QueSTS model also requires keywords though, which are extracted from
	the keywords section of the research paper itself (if present).
	
	For the other sections, such as Introduction, related work etc, we follow the method as discussed in
	both \cite{brandon} and \cite{sravanthi}. We match the sentences in Introduction with abstract, and 
	select the ones with high similarity score.
	
	For the conclusion section of the paper, again we use a hybrid approach. We define query terms 
	such as "concludes", "presents" etc and from the title, and generate the slides as per 
	Algorithm 1 given in \cite{sravanthi}.
	
\subsubsection*{Adding Graphics}
	We include the inline graphic elements (such as equations) that are a part of the selected sentences.
	We also include all graphics that is referred to in a selected sentence (such as "refer equation 1.2")
	
	For other stand-alone images, we match the keywords with the caption, and include the image if the
	caption matches.
	
\subsection{Post Processing}
	Finally we use \LaTeX\ to render the slides. For each section we have at least one slide.
	Then we decide the number of slides based on the number of sentences selected, and all
	selected sentences appear as bulleted text.
	The final output is in the same order as the paper, and the graphical etc elements are
	nested in the sections they appear in.

{\small
\bibliographystyle{plain}
\bibliography{ref}
}

\end{document}
